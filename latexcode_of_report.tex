\documentclass{article}
\usepackage{ctex} % 支持中文
\usepackage{amsmath, amssymb} % 支持数学符号
\usepackage{geometry}
\usepackage{xcolor} % 支持颜色
\usepackage{listings} % 支持代码高亮

% 设置页面尺寸
\geometry{a4paper, margin=1in}

% 代码高亮设置
\lstset{
    language=Python,
    basicstyle=\ttfamily\footnotesize,
    keywordstyle=\color{blue},
    commentstyle=\color{gray},
    stringstyle=\color{red},
    showstringspaces=false,
    numbers=left,
    numberstyle=\tiny,
    stepnumber=1,
    numbersep=5pt,
    tabsize=4,
    breaklines=true,
    breakatwhitespace=false,
    backgroundcolor=\color{lightgray!20},
    frame=single,
    captionpos=b,
    keepspaces=true,
}

\begin{document}

% 封面页
\title{哈密顿问题}
\author{}
\date{}
\maketitle

\thispagestyle{empty}
\newpage

% 正文内容
\section*{周游世界问题}

\begin{itemize}
  \item 1859年,数学家哈密顿将正十二面体的每个顶点比作一个城市,连接两个顶点之间的边看作城市之间的交通线,提出周游世界问题:能否从某个城市出发沿交通线经过每个城市一次并且仅一次,最后回到出发点?
\end{itemize}



\section*{哈密顿图}

\begin{itemize}
  \item \textcolor{teal}{哈密顿通路}: 经过图中所有顶点的初级通路
  \item \textcolor{teal}{哈密顿回路}: 经过图中所有顶点的初级回路
  \item \textcolor{teal}{哈密顿图}: 有哈密顿回路的图
\end{itemize}



\section*{1. 必要条件}

\textbf{定理 ①}: 若 $G$ 是哈密顿图,则对于结点集 $V(G)$ 的任一非空真子集 $S \subset V(G)$ 有 $\omega(G - S) \leq |S|$。其中 $G - S$ 表示在 $G$ 中删去 $S$ 中的结点后所构成的图,$\omega(G - S)$ 表示 $G - S$ 的连通分支数。

\textbf{证明}: 设 $C$ 是 $G$ 的一条哈密顿回路,$C'$ 视为 $G$ 的子图,在回路 $C$ 中,每删去 $S$ 中的一个结点,最多增加一个连通分支,且删去 $S$ 中的第一个结点时分支数不变,所以有 $\omega(C - S) \leq |S|$。又因为 $C$ 是 $G$ 的生成子图,所以 $C - S$ 是 $G - S$ 的生成子图,且 $\omega(G - S) \leq \omega(C - S)$,因此 $\omega(G - S) \leq |S|$。

这不是充分条件,反例是彼得松图,这段代码通过暴力搜索验证了彼得松图不是哈密顿图。

\newpage

\section*{Python 代码验证}

\begin{lstlisting}
import networkx as nx
from itertools import permutations

def is_hamiltonian_cycle(graph, cycle):
    """
    检查给定的顶点排列是否构成哈密顿环
    :param graph: 待检查的图
    :param cycle: 顶点的排列
    :return: 如果是哈密顿环返回True,否则返回False
    """
    # 检查排列长度是否与顶点数相同
    if len(cycle) != len(graph.nodes):
        return False
    
    # 检查排列中的每对相邻顶点是否有边连接
    for i in range(len(cycle)):
        if not graph.has_edge(cycle[i], cycle[(i + 1) % len(cycle)]):
            return False
    
    return True

def main():
    # 生成彼得森图
    G = nx.petersen_graph()
    
    # 获取彼得森图的所有顶点
    nodes = list(G.nodes)
    
    # 尝试所有顶点排列,检查是否有哈密顿环
    for perm in permutations(nodes):
        if is_hamiltonian_cycle(G, perm):
            print("彼得森图有哈密顿环:", perm)
            return
    
    # 如果循环结束没有找到哈密顿环,输出彼得森图没有哈密顿环
    print("彼得森图没有哈密顿环")

if __name__ == "__main__":
    main()
\end{lstlisting}

\newpage

\section*{2.简单的充分条件}

设 $G$ 是 $n(n \geqslant 3)$ 阶简单图,且对每一对顶点 $v, v^{\prime}$ 有

\[ d(v) + d\left(v^{\prime}\right) \geqslant n - 1, \]

则图 $G$ 有哈密顿链。

\textbf{证明} 先证明 $G$ 是连通图。若 $G$ 有两个或两个以上的连通部分,设其中之一有 $n_{1}$ 个顶点,另一部分有 $n_{2}$ 个顶点。分别从中各取一顶点 $v_{1}$ 、 $v_{2}$ ,则 $d\left(v_{1}\right) \leqslant n_{1} - 1, d\left(v_{2}\right) \leqslant n_{2} - 1$ 。故

\[ d\left(v_{1}\right) + d\left(v_{2}\right) \leqslant n_{1} + n_{2} - 2 < n - 1, \]

这与题设矛盾,所以 $G$ 是连通图。

现证明存在哈密顿链。证明的方法实际上给出一种哈密顿链的构造步骤。设在 $G$ 中有一条从 $v_{1}$ 到 $v_{p}$ 的链: $v_{1} v_{2} \cdots v_{p}$ 。如果有 $v_{1}$ 或 $v_{p}$ 与不在这条链上的一个顶点相邻,我们可扩展这条链,使它包含这个顶点。否则, $v_{1}$ 和 $v_{p}$ 都只与这条链上的顶点相邻,这时存在一个圈包含顶点 $v_{1}, v_{2}, \cdots, v_{p}$ 。假设与 $v_{1}$ 点相邻的顶点集是 $\left\{ v_{j_{1}}, v_{j_{2}}, \cdots, v_{j_{k}} \right\}$ ,这里 $v_{j_{1}}, v_{j_{2}}, \cdots, v_{j_{k}}$ 都是链 $v_{1} v_{2} \cdots v_{p}$ 中的点,且 $p < n$ 。

如果 $v_{1}$ 与 $v_{p}$ 相邻,则显然存在一个圈 $v_{1} v_{2} \cdots v_{p} v_{1}$ 。

如果 $v_{1}$ 和 $v_{p}$ 不相邻,则必然存在一点 $v_{l} (2 \leqslant l \leqslant p)$ 和 $v_{1}$ 相邻,而 $v_{l-1}$ 和 $v_{p}$ 相邻 。因为否则 $v_{p}$ 最多只和 $p - k - 1$ 个顶点相邻,即排除 $v_{j_{1}-1}$ 、 $v_{j_{2}-1}$ 、 $\cdots$ 、 $v_{j_{k}-1}$ 和 $v_{p}$ 自身,这样

\[ d\left(v_{1}\right) + d\left(v_{p}\right) \leqslant k + (p - k - 1) = p - 1 < n - 1, \]

这与假设矛盾。因而存在 $v_{1}, v_{2}, \cdots, v_{p}$ 的圈 $v_{1} v_{l} v_{l+1} \cdots v_{p} v_{l-1} v_{l-2} \cdots v_{2} v_{1}$ 。
若 $p = n$,实际上已经存在一个哈密顿圈。若 $p < n$,由于 $G$ 是连通的,所以必定存在一个不属于此圈的顶点 $v'$ 与 $v_1 v_2 \cdots v_p$ 中的某个相邻。不断重复以上步骤直到得到一条n-1的链为止.

(Ore)设 $G$ 是 $n(n \geqslant 3)$ 阶简单图,且对每一对不相邻的顶点 $v, v'$ 有
\[ d(v) + d(v') \geqslant n, \]
那么图 $G$ 有哈密顿圈。

证明:当 $n = 3$ 时,由所给条件知 $G$ 一定是完全图 $K_3$,命题成立。设 $n \geqslant 4$,用反证法证明。设 $G$ 是有 $n$ 个顶点且满足度数条件却没有哈密顿圈的图。不妨设 $G$ 是具有这种性质的边数最大的图,也就是说 $G$ 添上一条边就具有哈密顿圈(否则 $G$ 可以添加一些边,直到不能再添为止,加边后顶点的度数条件仍满足),由此得出在图 $G$ 中有一条包含图中每一个顶点的哈密顿链,记为 $v_1 v_2 \cdots v_n$。则 $v_1$ 与 $v_n$ 不相邻,于是
在 $v_2, v_3, \cdots, v_{n-1}$ 中必有一点 $v_i$ 使 $v_1$ 与 $v_i$ 相邻,$v_n$ 与 $v_{i-1}$ 相邻,如图所示。否则,有 $d(v_1) = k$ 个点 $v_{i_1}, v_{i_2}, \cdots, v_{i_k} \left(2 \leqslant i_1 \leqslant i_2 \leqslant \cdots \leqslant i_k \leqslant n-1\right)$ 与 $v_1$ 相邻,而 $v_n$ 与 $v_{i_1-1}, v_{i_2-1}, \cdots, v_{i_k-1}$ 都不相邻,从而
\[ d(v_n) \leqslant n-1-k, \]
则
\[ d(v_1) + d(v_n) \leqslant k + n - 1 - k = n - 1 < n, \]
这与条件矛盾。故 $G$ 存在一条哈密顿圈 $v_1 v_2 \cdots v_{i-1} v_n v_{n-1} \cdots v_i v_1$。这又与假设矛盾,从而命题得证。

\section{Python验证代码}
\begin{lstlisting}
# -*- coding: utf-8 -*-
"""
Created on Tue Jun 11 21:06:35 2024

@author: 蔚元利
"""

import numpy as np
import networkx as nx
import matplotlib.pyplot as plt

def generate_ore_graph(n):
    """
    生成一个满足Ore性质的随机图
    """
    G = nx.Graph()
    G.add_nodes_from(range(n))
    
    # 随机生成边,确保满足Ore性质
    for u in range(n):
        for v in range(u + 1, n):
            if np.random.rand() < 0.5:
                G.add_edge(u, v)
    
    # 强制确保满足Ore性质
    for u in range(n):
        for v in range(u + 1, n):
            if not G.has_edge(u, v):
                if G.degree[u] + G.degree[v] >= n:
                    G.add_edge(u, v)
    
    return G

def find_hamiltonian_cycle(graph, n):
    def extend_path(path):
        while True:
            extended = False
            for v in range(n):
                if v not in path and graph.has_edge(path[-1], v):
                    path.append(v)
                    extended = True
                    break
            if not extended:
                break
        return path

    def find_cycle_from_path(path):
        y1, ym = path[0], path[-1]
        
        # 检查 y1 和 ym 是否相邻
        if graph.has_edge(y1, ym):
            if len(path) == n:
                return path
            else:
                return None
        
        # 找到一个不在 path 中的顶点 z,使得 z 与 y1 相邻
        z_candidates = [z for z in range(n) if z not in path and graph.has_edge(z, y1)]
        if z_candidates:
            z = z_candidates[0]
            path = [z] + path
            path = extend_path(path)
        else:
            # 找到一个顶点 yk,使得 y1 和 yk 邻接,并且 ym 和 yk-1 邻接
            for k in range(1, len(path)):
                if graph.has_edge(y1, path[k]) and graph.has_edge(ym, path[k - 1]):
                    path = path[:k] + [ym] + path[k:]
                    path = extend_path([y1] + path)
                    break
            else:
                return None
        
        return find_cycle_from_path(path)
    
    path = extend_path([0])
    cycle = find_cycle_from_path(path)
    
    return cycle

def main():
    # 用户输入图的顶点数
    n = int(input("请输入顶点数(n > 3):"))
    if n <= 3:
        print("顶点数必须大于3")
        return
    
    # 生成满足Ore性质的随机图
    while True:
        graph = generate_ore_graph(n)
        if all(graph.degree[u] + graph.degree[v] >= n for u in range(n) for v in range(u + 1, n) if not graph.has_edge(u, v)):
            break
    
    # 查找哈密顿链
    cycle = find_hamiltonian_cycle(graph, n)
    
    if cycle:
        print("找到哈密顿链:", cycle)
    else:
        print("未找到哈密顿链")
    
    # 绘制图形
    pos = nx.spring_layout(graph)
    nx.draw(graph, pos, with_labels=True, node_color='lightblue', edge_color='gray')
    if cycle:
        edges = [(cycle[i], cycle[(i+1)%n]) for i in range(len(cycle))]
        nx.draw_networkx_edges(graph, pos, edgelist=edges, edge_color='r', width=2)
    plt.show()

if __name__ == "__main__":
    main()
\end{lstlisting}
\section*{3. 对于平面图的必要条件}
如果一个平面图有哈密顿圈 $c$, 用 $f_i^{\prime}$ 表示在 $c$ 的内部的 $i$ 边形的个数, 用 $f_i^{\prime \prime}$ 表示在 $c$ 的外部的 $i$ 边形的个数, 则
(1) $1 \cdot f_3^{\prime}+2 \cdot f_4^{\prime}+3 \cdot f_5^{\prime}+\cdots=n-2$;
(2) $1 \cdot f_3^{\prime \prime}+2 \cdot f_4^{\prime \prime}+3 \cdot f_5^{\prime \prime}+\cdots=n-2$;
(3) $1 \cdot\left(f_3^{\prime}-f_3^{\prime \prime}\right)+2 \cdot\left(f_4^{\prime}-f_4^{\prime \prime}\right)+3 \cdot\left(f_5^{\prime}-f_5^{\prime \prime}\right)+\cdots=0$.

其中 $n$ 为 $G$ 的顶点数, 显然也是 $c$ 的长.
证明 设 $c$ 的内部有 $d$ 条边. 由于 $G$ 是平面图, 它的边都不相交, 所以一条边把它经过的面分成两部分. 设想这些边是一条一条地放进图里去的, 每面. 于是 $c$ 的内部的面的总数为
$$
f_2^{\prime}+f_3^{\prime}+f_4^{\prime}+f_5^{\prime}+\cdots=d+1 .
$$

在 $c$ 内每个 $i$ 边形中记上数字 $i$, 各面所记数字之和就是围成这些面的边的总数, $c$ 内部的每一条边都被数了两次, 而 $c$ 上的 $n$ 条边, 每条边都只数了一次,于是
$$
2 f_2^{\prime}+3 f_3^{\prime}+4 f_4^{\prime}+5 f_5^{\prime}+\cdots=2 d+n .
$$
(2)式减去(1)式的两倍, 得
$$
1 \cdot f_3^{\prime}+2 \cdot f_4^{\prime}+3 \cdot f_5^{\prime}+\cdots=n-2 .
$$

类似地可以推得
$$
1 \cdot f_3^{\prime \prime}+2 \cdot f_4^{\prime \prime}+3 \cdot f_5^{\prime \prime}+\cdots=n-2 .
$$
(3)、(4)两式相减即得
\[ 1 \cdot \left( f_3' - f_3'' \right) + 2 \cdot \left( f_4' - f_4'' \right) + 3 \cdot \left( f_5' - f_5'' \right) + \cdots = 0 \]
\section{Python验证代码}
\begin{lstlisting}
import networkx as nx
import matplotlib.pyplot as plt
from itertools import permutations

def generate_random_planar_graph(num_nodes):
    """
    生成一个随机平面图
    :param num_nodes: 节点数
    :return: 平面图
    """
    G = nx.erdos_renyi_graph(num_nodes, 0.4)
    while not nx.check_planarity(G)[0]:
        G = nx.erdos_renyi_graph(num_nodes, 0.4)
    pos = nx.planar_layout(G)
    return G, pos

def is_hamiltonian_cycle(graph, cycle):
    """
    检查给定的顶点排列是否构成哈密顿环
    :param graph: 待检查的图
    :param cycle: 顶点的排列
    :return: 如果是哈密顿环返回True,否则返回False
    """
    if len(cycle) != len(graph.nodes):
        return False

    for i in range(len(cycle)):
        if not graph.has_edge(cycle[i], cycle[(i + 1) % len(cycle)]):
            return False

    return True

def find_hamiltonian_cycle(graph):
    """
    尝试找到图中的哈密顿环
    :param graph: 待检查的图
    :return: 如果找到哈密顿环,返回环中的节点顺序,否则返回None
    """
    nodes = list(graph.nodes)
    for perm in permutations(nodes):
        if is_hamiltonian_cycle(graph, perm):
            return perm
    return None

def verify_hamiltonian_theorem(graph, cycle):
    """
    根据定理验证一个平面图是否是哈密顿图
    :param graph: 待检查的图
    :param cycle: 哈密顿环
    :return: True如果满足定理条件, 否则False
    """
    if not cycle:
        return False

    n = len(graph.nodes)
    f_prime = [0] * (n + 1)
    f_double_prime = [0] * (n + 1)

    # 获取所有的面
    faces = nx.cycle_basis(graph)
    internal_faces = []
    external_faces = []

    # 根据哈密顿环将面分为内部和外部
    for face in faces:
        if all(node in cycle for node in face):
            internal_faces.append(face)
        else:
            external_faces.append(face)

    for face in internal_faces:
        f_prime[len(face)] += 1

    for face in external_faces:
        f_double_prime[len(face)] += 1

    # 计算内部和外部的定理值
    sum_internal = sum(i * f_prime[i] for i in range(3, n + 1))
    sum_external = sum(i * f_double_prime[i] for i in range(3, n + 1))

    return sum_internal == n - 2 and sum_external == n - 2

def main():
    num_nodes = int(input("请输入平面图的节点数: "))
    G, pos = generate_random_planar_graph(num_nodes)

    # 找到哈密顿环
    cycle = find_hamiltonian_cycle(G)

    if cycle:
        print("找到哈密顿环:", cycle)
        # 验证定理
        if verify_hamiltonian_theorem(G, cycle):
            print("根据定理,该图是哈密顿图")
        else:
            print("根据定理,该图不是哈密顿图")
    else:
        print("没有找到哈密顿环,该图不是哈密顿图")

    # 绘制图
    nx.draw(G, pos, with_labels=True)
    plt.show()

if __name__ == "__main__":
    main()
\end{lstlisting}
\section*{4.邦迪定理和邦迪-萨瓦达定理}
\textbf{Bondy's Theorem (1972)}

设 \( G \) 为 \( n \) 阶简单无向图,且 \( n \geq 3 \),则图 \( G \) 是Hamilton图当且仅当图 \( G \) 的闭包 \(\mathrm{Cl}(G)\) 是Hamilton图.

\textbf{闭包}: 图 \( G \) 的闭包 \(\mathrm{Cl}(G)\) 是指由下述方法得到的一个图,即反复连接 \( G \) 中度数之和 \(\geq n\) 的不相邻的两顶点,直到没有这样的顶点对存在为止.

\textbf{Proof}:

\textbf{引理}: 图 \( G \) 的闭包 \(\mathrm{Cl}(G)\) 是唯一确定的.

\textbf{证明}: 假设 \( G \) 有两个闭包 \( G_{1} \) 和 \( G_{2} \),设 \(\{e_{i}\}_{i=1}^{m}\) 是经由上述操作所添加的边, \(\{f_{i}\}_{i=1}^{k}\) 是 \( G_{2} \) 所添加的边。令 \( H = G + \{e_{i}\}_{i=1}^{m}\),不妨设 \( e_{k+1} = uv \) 是 \(\{e_{i}\}\) 中第一条不属于 \( G_{2} \) 的边,则 \( H \subseteq G_{2} \),故
\[ 
d_{G_{2}}(u) + d_{G_{2}}(v) \geq n 
\]
由 \( G_{1} \) 的定义可得
\[ 
d_{H}(u) + d_{H}(v) \geq n 
\]
矛盾!

因此每条 \( e_{i} \) 都是 \( G_{2} \) 的边,同理可得每条 \( f_{i} \) 都是 \( G_{1} \) 的边,故 \( G_{1} = G_{2} \),即 \( G \) 的闭包 \(\mathrm{Cl}(G)\) 是唯一确定的.

下面证明 \( G \) 是Hamilton图。在构作闭包的过程中,每添加一条边都用一次Ore定理即可 。证毕。




\textbf{推论}: 若 \( \mathrm{Cl}(G) \cong K_{n} \) (\( n \geq 3 \)),则 \( G \) 是Hamilton图。

显然 \( K_{n} \) 是Hamilton图,它的最外圈 \( C_{n} \) 即为其Hamilton回路,故由Bondy定理得证。

\textbf{Bondy-Chvátal Theorem (1976)}

设 \( G \) 为 \( n \) 阶简单无向图,且 \( n \geq 3 \)。设 \( G \) 的度序列为 \( \left\{d_{i}\right\}_{i=1}^{n}, n \geq 3 \),且 \( d_{1} \leq d_{2} \leq \cdots \leq d_{n} \)。若不存在 \( m \leq\left\lfloor\frac{n}{2}\right\rfloor \) ,使得 \( d_{m} \leq m \wedge d_{n-m} \leq n-m \),则 \( G \) 是Hamilton图。

\textbf{Proof}:

我们先来证明 \( \mathrm{Cl}(G) \cong K_{n} \)。

假设 \( \mathrm{Cl}(G) \nsubseteq K_{n} \),设 \( u, v \) 是 \( \mathrm{Cl}(G) \) 中两个不相邻的顶点,有

\[ d_{\mathrm{Cl}}(u) \leq d_{\mathrm{Cl}}(v) \]

且 \( d_{\mathrm{Cl}}(u)+d_{\mathrm{Cl}}(v) \) 充分大,并且有

\[ \forall u, v \in \mathrm{Cl}(G) \wedge u v \notin \mathrm{Cl}(G) \text { s.t. } d_{\mathrm{Cl}}(u)+d_{\mathrm{Cl}}(v)<v \]

记

\[
\begin{array}{l}
S=\left\{v_{i} \mid v_{i} \in \mathrm{Cl}(G) \wedge v_{i} v \notin \mathrm{Cl}(G)\right\} \\
T=\left\{v_{i} \mid v_{i} \in \mathrm{Cl}(G) \wedge v_{i} u \notin \mathrm{Cl}(G)\right\}
\end{array}
\]

则有

\[
\begin{array}{l}
|S|=n-1-d_{\mathrm{Cl}}(v) \\
|T|=n-1-d_{\mathrm{Cl}}(u)
\end{array}
\]

由于 \( \Delta(S) \leq d_{\mathrm{Cl}}(u), \Delta(T \cup\{u\}) \leq d_{\mathrm{Cl}}(v) \),令 \( d_{\mathrm{Cl}}(u)=m \),则 \( V(\mathrm{Cl}(G)) \geq m, \Delta(\mathrm{Cl}(G)) \leq m \) 且 \( \left|\left\{v_{i} \mid d_{\mathrm{Cl}}\left(v_{i}\right) \leq n-m\right\}\right| \geq n-m \)。因为 \( G \) 是 \( \mathrm{Cl}(G) \) 的生成子图,所以上述结论对 \( G \) 也成立,即
\( d_{m} \leq m, d_{n-m} \leq n-m \),又由 \( (1) \) \( (2) \) 可知 \( m<\frac{n}{2} \),这与 \( \mathrm{Cl}(G) \nsubseteq K_{n} \) 矛盾!

故 \( \mathrm{Cl}(G) \cong K_{n} \)。再由Bondy定理的推论可得 \( G \) 是Hamilton图。
\section{5.posa定理}


\textbf{Pósa's Theorem (1962)}

设 \( G \) 为 \( n \) 阶简单无向图,且 \( n \geq 3 \)。定义 \( V_{k}:=\{v \in V: \operatorname{deg}(v) \leq k\} \),若对于每一个 \( k \in \mathbb{Z}^{+} \) 满足 \( 1 \leq k < \frac{n-1}{2} \), \( \left|V_{k}\right| < k \),且当 \( n \) 为奇数时满足 \( k=\frac{n-1}{2}\), \( \left|V_{k}\right| \leq k \) 则 \( G \) 是Hamilton图。

\textbf{证明}

我们首先来证明 \( \mathrm{Cl}(G) \cong K_{n} \)。假设 \( \mathrm{Cl}(G) \nsubseteq K_{n} \),取 \( \mathrm{Cl}(G) \) 中不相邻的两顶点 \( u, v \) 使得 \( d_{\mathrm{Cl}}(u)+d_{\mathrm{Cl}}(v) \) 最大,则有 \( d_{\mathrm{Cl}}(u)+d_{\mathrm{Cl}}(v) \leq n-1 \)。我们不妨设 \( d_{\mathrm{Cl}}(u) \leq d_{\mathrm{Cl}}(v) \),设 \( d_{\mathrm{Cl}}(u)=k \leq \frac{n-1}{2} \),则有 \( d_{\mathrm{Cl}}(v) \leq n-k-1\)。设 \( S=\{x \in \mathrm{Cl}(G): x v \notin \mathrm{Cl}(G), x \neq v\}\),所以 \( u \in S \)。若 \( w \in S \),则 \( d_{\mathrm{Cl}}(x) \leq k\),否则

\[ d_{\mathrm{Cl}}(w)+d_{\mathrm{Cl}}(v) > d_{\mathrm{Cl}}(u)+d_{\mathrm{Cl}}(v) \]

这与 \( u, v \) 的定义矛盾。因此 \( \Delta(S) \leq k, |S| \leq k-1 \),则

\[ d_{\mathrm{Cl}}(v) \geq (n-1)-(k-1) = n-k \]

这就产生了矛盾。故 \( \mathrm{Cl}(G) \cong K_{n} \)。由Bondy定理可知 \( G \) 显然为 Hamilton图。

\end{document}



\end{document}
